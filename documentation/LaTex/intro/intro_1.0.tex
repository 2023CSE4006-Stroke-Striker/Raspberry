\section{Introduction}
\subsection{Motivation}
\begin{itemize}
    \item Problem: Unannounced emergencies, poor initial response \\
    \cite{r8}According to the National Statistical Office of the Republic of Korea, nearly 60,000 of the total 120,000 stroke patients in 2021 were not transferred to the emergency room until more than six hours after the outbreak.
Fewer than 15\% of the people arrived at the emergency room in less than an hour, and half of them were patients living in the Seoul-Gyeonggi area.
\cite{r9} The number of patients at risk increases in the provinces, but the medical infrastructure is insufficient, so initial diagnosis or prevention is not possible, and even if it could be, follow-up will inevitably be delayed.
After all, time is the lifeblood of a stroke. In order to do that, you need to quickly notice the signs. But it's rare to check for signs of stroke "every day".
We think the capture in everyday life is key. To do this, we'll have to mount active detectors on the ones we face most frequently.
Additionally, it is not just a project for the elderly. \cite{r8} According to the National Medical Center, the incidence of stroke among people in their 20s and 30s is rising every year.
Because of the stereotype that stroke is the disease of the elderly, even if you have premonitory symptoms, you will often pass it on, thinking of it as another reason.
\cite{r10} As the proportion of single-person households in Korea approaches 34.5\%, chances for them to recognize acute diseases in the early stages are also decreasing. 
In order to become a better society, we must overcome this.

The reality is that people's recognition rate of early stroke symptoms is also very low.
\cite{r11} According to the Korea Centers for Disease Control and Prevention, only 54\% of all respondents were correct for early stroke symptoms.
Although awareness reached a high of 61\% during the pandemic in 2019 - presumably because people cared a lot about health issues due to pandemic.
It has been low since 2019.
Even though we are living in an information society, this figure is that we don't usually think about stroke, and now you can see that why the effectiveness of passive stroke detection is very low.

The problems we thought of are summarized as follows.

First, people aren't as wary or concerned about strokes as we might think.
Second, when a stroke occurs, it is quite rare to arrive at a hospital within an hour, the golden time, and if it can, most of them live in the metropolitan area with medical infrastructure that allows immediate emergency action.
Third, the elderly, who are at high risk of getting the disease, are concentrated in the provinces where have no infrastructure, and have low awareness, so it is unlikely to respond to premonitory symptoms.
Fourth, although a situation where the incidence rate in their 20s and 30s is rising, their vigilance is very low.
Fifth, as the number of single-person households increases, there are fewer opportunities for them to recognize and initial responses are becoming insufficient.

In order for the passive detector to be effective, people of all ages must be aware of stroke on their own and check it periodically. So the effect of it does not seem to be able to be enhanced by any method-promotion, campaign, etc. -If this could be elevated, We think \cite{r12}the prognostic indicator for stroke should have been more positive.

Therefore, what we need is an "active daily checker", which we think can be made through home appliances, typically refrigerators or TVs.
    
\end{itemize}
\\
\begin{itemize}
    \item Solve\\
    As I said at the beginning, \cite{r7} We open the refrigerator once a day for a reason or not. Our concept uses this habit.
There are already many refrigerators equipped with IoT technology. We're going to turn up our home appliances by locating a camera module here.
The refrigerator detects the person's face and its landmark through computer vision when a person stands in front. 
Among BE-FAST diagnostic methods, if facial expressions such as paralysis of one facial muscle, are detected, stroke-striker notify it right away.
The notification method is that if you have a refrigerator with a display, it can tell you through the display, and if you have not, it can tell you through a push-message on your phone application (Thin-Q) or use AI speaker (NUGU).
At a later stage, it will guide you to the nearest hospital where first aid for stroke is available, and automatically connect to 119 if the user wants.
Since fixed cameras may not respond appropriately depending on the user's physical characteristics, \cite{r13} multi-angle vision technology is applied to detect users from various angles.
This creates a true daily active detector beyond the limits of different physical and home structural characteristics for each user.
\end{itemize}
\\
\begin{itemize}
    \item Expectation \\
    As the artificial intelligence field is rapidly developing and computer vision is a technology that occupies a large proportion of it, it is highly likely to detect user behavior or face and develop it into various medical diagnosis.
If these technologies are included in each of LG Electronics' home appliances, which currently have a huge share of home appliances compared to competitors, users will be able to continue to actively observe them in their homes, whether in living rooms, kitchens or bedrooms.
This will allow the home to become a diagnostic center for individuals that move actively beyond just living spaces, and if this becomes a reality, we expect a very big paradigm shift.
\cite{r14}We think home diagnostics self care, which is developing recently, is a very important technology field, and the synergy will be great if it is combined with home appliances.
\end{itemize}

\subsection{Research on Related Materials}
\begin{itemize}
    \item Project MONAI \\
    MONAI is an initiative started by NVIDIA and King's College London to establish an inclusive community of AI researchers to develop and exchange best practices for AI in healthcare. This collaboration has expanded to include academic and industry leaders throughout the medical field. \\
    This project is similar to our project because it is simply analyzing MRI or CT photographs with AI, but the methods used are different.
    \item BASLER \\
    % 비전 시스템의 전반적인 솔루션을 제공한다. 하드웨어와 소프트웨어를 동시에 지원하고 머신러닝 기반으로 이미지를 판별할 수 있다. 특히, 의료 관련에 특화되어 있다. 하지만 센서와 카메라가 매우 비싸 그대로 가전에 적용하기는 어렵다.
    This project provides an overall solution for the vision system. It supports hardware and software at the same time and can analyse images based on machine learning. It is specialized in medical care particularly. However, sensors and cameras are very expensive, so it would be difficult to apply them to home appliances as they are presented in this project.
    \item Kaggle Project \\
    % kaggle에서 진행하는 뇌졸중 감지 프로젝트이다. 우리 프로젝트의 AI 모델로써 사용가능하다. 이 프로젝트의 판별 알고리즘은 2D 이미지를 기반으로 진행하기 때문에 우리 프로젝트의 3D 인식과는 다른 점이 있다.
    It is a stroke detection project undertaken by Kaggle. It can be used as an AI model for our project but since the algorithm used in this project is based on 2D images, it differs from the 3D recognition we need to use in our project.
    \item Related Papers \\
    % 우리의 프로젝트의 이론적인 근거를 찾기 위해 많은 논문을 조사했다. 조사한 결과 주요한 논문은 다음과 같다. FAST 기법으로 얼굴 사진으로 뇌졸중을 판단할 수 있고, 하나의 카메라로 multi-angle의 얼굴 표정을 검출할 수 있다. 마지막으로 CNN을 이용해 facial landmark로 딥러닝을 수행할 수 있음을 알 수 있었다.
    We researched a number of papers to find the theoretical part for our project.\\ 
    % The FAST technique can determine a stroke with a facial photo.Then, a multi-angle facial expression can be detected with one camera. Finally, we found that deep learning can be performed with facial landmark using CNN.

    1. Multi-Angle detector\cite{r15}
    \\
    \\This paper introduce lightweight deep network and combining key point feature positioning for multi-angle face expression recognition. Using robot dog to recognize facial expressions will be affected by distance and angle. To solve this problem, this paper proposes a method for facial expression recognition at different distances and angles,which solved the larger distance and deflection angle of face expression recognition \\

    2. Raspberry Pi Based Emotion Recognition using OpenCV, TensorFlow, and Keras \cite{r16}
    \\
    \\It implement an Emotion Recognition System or a Facial Expression Recognition System on a Raspberry Pi 4. It apply a pre-trained model to recognize the facial expression of a person from a real-time video stream. The “FER2013” dataset is used to train the model with the help of a VGG-like Convolutional Neural Network (CNN).\\

    3. Connect a Raspberry Pi or other device with AWS\cite{r17} \\
    \\It tells how to Set up device, install the required tools and libraries for the AWS IoT Device SDK, install AWS IoT Device SDK, install and run the sample appView messages from the sample app in the AWS IoT console.\\

    4. Realtime Facial Emotion Recognition\cite{r18} \\
    \\This repository demonstrates an end-to-end pipeline for real-time Facial emotion recognition application through full-stack development. The front-end is developed in react.js and the back-end is developed in FastAPI. The emotion prediction model is built with Tensorflow Keras, and for real-time face detection with animation on the frontend, Tensorflow.js have been used.\\

    5. Kaggle FER-2013 DataSet\cite{r19} \\
    \\The data consists of 48x48 pixel grayscale images of faces. The faces have been automatically registered so that the face is more or less centred and occupies about the same amount of space in each image.
The task is to categorize each face based on the emotion shown in the facial expression into one of seven categories (0=Angry, 1=Disgust, 2=Fear, 3=Happy, 4=Sad, 5=Surprise, 6=Neutral). The training set consists of 28,709 examples and the public test set consists of 3,589 examples.\\

    6. Facial landmarks with dlib, OpenCV, and Python\cite{r20} \\
    \\This document tell What are face landmarks, Understanding dlib’s facial landmark detector, How Detect facial landmarks with dlib, OpenCV, and Python,
    how visualization facail landmarks with co-lab,Google.
    Also, it introduce alternative facial landmark detectors, such as MediaPipe library which is capable of computing a 3D face mesh.\\
    
\end{itemize}
