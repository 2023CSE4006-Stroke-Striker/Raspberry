\section{\textbf{Introduction}}
\subsection{\textbf{Motivation}\\}
\begin{itemize}
    \item \textbf{The Problem}\\
    \\ \cite{r8} According to the National Statistical Office of the Republic of Korea, nearly 60,000 of the total 120,000 stroke patients in 2021 were not transferred to the emergency room until more than six hours after the outbreak. Fewer than 15\% of the people arrived at the emergency room in less than an hour, and half of them were patients living in the Seoul-Gyeonggi area. \cite{r9} The number of patients at risk is increasing in provinces and the medical infrastructure is insufficient, so initial diagnosis or prevention is not possible, and even if it could be, follow-up will inevitably be delayed. After all, time is the lifeblood of a stroke and every minute counts. You need to quickly notice the signs but rarely do patients check signs of stroke daily.\\ \\With the development of technology and healthcare in the world, life expectancy has gradually increased over the years and is approaching a staggering 80 years old. Health is one of the most important factors in a person's life. However, there are cases where the elderly are reluctant to go to the hospital due to their habits or due to their misunderstandings arising from their experiences. There are cases in which the right time to prevent or treat the person had already passed and it was already too late. Then, in this context, not worrying about one’s health and not taking preemptive measures can lead to serious social problems.\\ \\This problem is not only defined to the older generation. \cite{r8} According to the National Medical Center, the incidence of stroke among people in their 20s and 30s is rising every year. The stereotype that stroke is a disease only the elderly can have can often push people to neglect it even when having premonitory symptoms.\\ \\Moreover, a growing number of people have started living alone, especially in South Korea, and similar problems can appear in these single-person households. Because of living alone, ones can’t point out their unhealthy habits. And if this person gets an acute disease, they won’t be able to take the proper measures leading to serious health problems. \cite{r10} As the proportion of single-person households in Korea approaches 34.5\%, chances for them to recognize acute diseases in the early stages are also decreasing.\\ \\In order to become a better society, we must overcome all those problems. The reality is that people's recognition rate of early stroke symptoms is very low. \cite{r11} According to the Korea Centers for Disease Control and Prevention, only 54\% of all respondents were correct for early stroke symptoms. Although awareness reached a high of 61\% during the pandemic in 2019 - presumably because people cared a lot about health issues due to pandemic - it has been low since 2019.\\ \\Let's summarize everything. First, people aren't as wary or concerned about strokes as we might think. Second, when a stroke occurs, it is quite rare for patients to arrive at the hospital within the recommended time, and most of them actually live in the metropolitan area where the medical infrastructure doesn't meet quality requirements. Third, the elderly, who are at high risk of getting the disease, are concentrated in provinces with low-quality healthcare institutions, and have low awareness about the disease so it is unlikely for them to respond to premonitory symptoms. Fourth, although the incidence rate of stroke for people in their 20s and 30s is rising, their vigilance is still very low. Fifth, as the number of single-person households is increasing, ones are not able to detect stroke beforehand and initial responses are becoming insufficient.\\ \\In order for the passive detectors to be effective, people of all ages must be aware of stroke on their own and check it periodically. However, no method, from promotions to campaigns, seems to be able to enhance this phenomenon. If this was the case, \cite{r12} the prognostic indicator for stroke should have been higher.\\ \\Therefore, what we need is an active, every day, stroke checker.\\
\end{itemize}
\begin{itemize}
    \item \textbf{The Solution}\\
    \\As we said at the beginning, \cite{r7} people open their refrigerator approximately once a day. Our concept benefits from this habit. There are already many refrigerators equipped with IoT technology, so our idea is to equip our home appliances with cameras. The refrigerator detects the person's face and his landmark through computer vision when he stands in front of it.\\ \\Using BE-FAST diagnostic methods, if a specific facial expression such as paralysis of one facial muscle is detected, "Strike Stroke" will notify the user right away. The notification method would be through a push-message on your phone application (Thin-Q) or by the use of an AI speaker (NUGU).\\ \\Afterwards, it will guide you to the nearest hospital or emergency center where first aid for stroke is available. It will offer users automatic connection to 119 if they approve it.\\ \\Since fixed cameras may not respond appropriately depending on the users' physical characteristics, \cite{r13} multi-angle vision technology is applied to detect them from various angles. This creates a true daily active detector, beyond the limits of difference in home structure and physical characteristics of each user.\\
\end{itemize}
\begin{itemize}
    \item \textbf{Future Expectations}\\
    \\As the artificial intelligence field is rapidly developing and computer vision is a technology that occupies a large proportion of it, it is highly likely to detect user behavior and develop it into various medical diagnosis. If these technologies are included in each of LG Electronics' home appliances, which currently have a huge share compared to other competitors, users will be able to continue to actively detect their diseases in their homes, whether in the living room, the kitchen or even the bedroom. This will allow the home to become an active diagnostic center for individuals rather than just passive living space. If this one day becomes our reality, we expect a very big paradigm shift. \cite{r14} We think home diagnostics self care, which is developing recently, is a very important technology field, and the synergy will be great if it is combined with home appliances.\\
\end{itemize}

\subsection{\textbf{Research on Related Materials}\\}
\begin{itemize}
    \item \textbf{Project MONAI}\\
    
    \begin{figure}[htp]
    \centering
    \includegraphics[width=4cm]{images/monai.png}
    \caption{MONAI project}
    \label{fig:monai}
    \end{figure}
    
    MONAI is an initiative started by NVIDIA and King's College London to establish an inclusive community of AI researchers to develop and exchange best practices for AI in healthcare. This collaboration has expanded to include academic and industry leaders throughout the medical field.\\ \\This project is similar to our project because it is simply analyzing MRI or CT photographs with AI, but the methods used are different.\\
    \item \textbf{BASLER}\\
    % 비전 시스템의 전반적인 솔루션을 제공한다. 하드웨어와 소프트웨어를 동시에 지원하고 머신러닝 기반으로 이미지를 판별할 수 있다. 특히, 의료 관련에 특화되어 있다. 하지만 센서와 카메라가 매우 비싸 그대로 가전에 적용하기는 어렵다.
    \\This company actually provides an overall solution for the vision system. Their products support hardware and software at the same time and can analyse images based on machine learning. However, their cameras and sensors are very expensive, so it would be difficult to apply them to home appliances as they are presented in this project.\\
    \item \textbf{Kaggle Project}\\

    \begin{figure}[htp]
    \centering
    \includegraphics[width=3cm]{images/kaggle.png}
    \caption{Kaggle}
    \label{fig:kaggle}
    \end{figure}
    
    % kaggle에서 진행하는 뇌졸중 감지 프로젝트이다. 우리 프로젝트의 AI 모델로써 사용가능하다. 이 프로젝트의 판별 알고리즘은 2D 이미지를 기반으로 진행하기 때문에 우리 프로젝트의 3D 인식과는 다른 점이 있다.
    It is a stroke detection project undertaken by Kaggle. It can be used as an AI model for our project but since the algorithm used in this project is based on 2D images, it differs from the 3D recognition we need to use in our project.\\
    \item \textbf{Related Papers}\\
    % 우리의 프로젝트의 이론적인 근거를 찾기 위해 많은 논문을 조사했다. 조사한 결과 주요한 논문은 다음과 같다. FAST 기법으로 얼굴 사진으로 뇌졸중을 판단할 수 있고, 하나의 카메라로 multi-angle의 얼굴 표정을 검출할 수 있다. 마지막으로 CNN을 이용해 facial landmark로 딥러닝을 수행할 수 있음을 알 수 있었다.
    \\We researched numerous papers in order to study the theoretical part of our project.\\ 
    % The FAST technique can determine a stroke with a facial photo.Then, a multi-angle facial expression can be detected with one camera. Finally, we found that deep learning can be performed with facial landmark using CNN.

    1. Multi-Angle detector\cite{r15}\\
    \\This paper introduces lightweight deep network and combining key point feature positioning for multi-angle facial expression recognition. Using robot dog to recognize facial expressions will be affected by distance and angle. To solve this problem, this paper proposes a method for facial expression recognition at different distances and angles, which solved the larger distance and deflection angle of facial expression recognition accuracy and real-time issues.\\

    2. Raspberry Pi Based Emotion Recognition using OpenCV, TensorFlow, and Keras \cite{r16}\\
    \\In this tutorial, they implement an Emotion Recognition System or a Facial Expression Recognition System on a Raspberry Pi 4. The apply a pre-trained model in order to recognize the facial expression of a person from a real-time video stream. The “FER2013” dataset is used to train the model with the help of a VGG-like Convolutional Neural Network (CNN).\\

    3. Connect a Raspberry Pi or other device with AWS\cite{r17} \\
    \\This step-by-step tutorial guides through all the steps you need to take in order to connect a Raspberry Pi or any other device with AWS. It tells you how to set up the device, install the required tools and libraries for the AWS IoT Device SDK, install AWS IoT Device SDK, install and run the sample app, as well as view the messages from the sample app in the AWS IoT console.\\

    4. Realtime Facial Emotion Recognition\cite{r18} \\
    \\This repository demonstrates an end-to-end pipeline for real-time Facial emotion recognition application through full-stack development. The front-end is developed in react.js and the back-end is developed in FastAPI. The emotion prediction model is built with Tensorflow Keras, and for real-time face detection with animation on the front-end, Tensorflow.js have been used.\\

    5. Kaggle FER-2013 DataSet\cite{r19} \\
    \\The data consists of 48x48 pixel grayscale images of faces. The faces have been automatically registered so that the face is more or less centred and occupies about the same amount of space in each image.\\ \\The task is to categorize each face based on the emotion shown in the facial expression into one of seven categories (0=Angry, 1=Disgust, 2=Fear, 3=Happy, 4=Sad, 5=Surprise, 6=Neutral). The training set consists of 28,709 examples and the public test set consists of 3,589 examples.\\

    6. Facial landmarks with dlib, OpenCV, and Python\cite{r20} \\
    \\This post explains line by line a source code and demonstrates in details what are face landmarks and how to detect facial landmarks using dlib, OpenCV, and Python. Also, it introduces alternative facial landmark detectors such as ones coming from the MediaPipe library which is capable of computing a 3D face mesh.\\
\end{itemize}