\documentclass[conference]{IEEEtran}
\IEEEoverridecommandlockouts
% The preceding line is only needed to identify funding in the first footnote. If that is unneeded, please comment it out.
\usepackage{cite}
\usepackage{amsmath,amssymb,amsfonts}
\usepackage{algorithmic}
\usepackage{graphicx}
\usepackage{textcomp}
\usepackage{kotex}
\usepackage{xcolor}
\usepackage{url}
\def\BibTeX{{\rm B\kern-.05em{\sc i\kern-.025em b}\kern-.08em
    T\kern-.1667em\lower.7ex\hbox{E}\kern-.125emX}}


\begin{document}

\title{Stroke Striker\\}

\author{\IEEEauthorblockN{Lee Seungsu}
\IEEEauthorblockA{\textit{dept. Computer Science} \\
Korea \\
2019034702 \\
mqm0051@gmail.com}
\and
\IEEEauthorblockN{Park Geonryul}
\IEEEauthorblockA{\textit{dept. Computer Science} \\
 Korea \\
2019040564 \\
geonryul0131@gmail.com}
\and
\IEEEauthorblockN{Elia Ayoub}
\IEEEauthorblockA{\textit{dept. Computer Science} \\
France \\
9170420231 \\
elia-ayoub@outlook.com}
\and
\IEEEauthorblockN{Ryan Jabbour}
\IEEEauthorblockA{\textit{dept. Computer Science} \\
France \\
9191820235 \\
jabbourryan2@gmail.com}
}

\maketitle

% %\begin{abstract}
% With the development of technology in the world, the average life expectancy of humans is gradually increasing worldwide.
% This means that old people would have to live a healthy life for a long time.
% However, elderly people get reluctant to go to the hospital, and if they do, they get examined at the hospital only when symptoms develop even though they are more vulnerable to diseases.
% We are going to present to you a new home appliance that can actively examine one's health, advice him, and warn him about suspicious symptoms correlated to dangerous diseases while keeping a diagnostic report to the user.
% \end{abstract}

\begin{abstract}
    Our model is an "active" stroke detector-strike stroke.

Stroke is a representative acute severe disease that ranks as the \cite{r1} fourth cause of death in Korea and \cite{r2}the fifth cause of death in the United States. The sooner the better, but \cite{r3} It is most effective to remove blood clots within an hour, and if not, treatment should be done within up to 3 hours to reduce side effects. \\
\\However, since CT, MRI scans, and diagnosis time after arriving at the hospital should be considered, \cite{r4} The golden time for stroke should be about an hour before arriving at the hospital.
However, the time of stroke after the onset of prognostic symptoms cannot be predicted, so if you see prognostic symptoms, you should go to the hospital right away.
\\Fortunately, \cite{r5} stroke has a reliable pre-hospital diagnostic method called BE-FAST (Balance, Eyes, Face, Arm, Speech, Terrible headache). 
Among them, Face-facial expression changes caused by paralysis of facial muscles are very obvious, so you can make a relatively accurate diagnosis.
\\
\\We focus on that.
\\Of course, there are currently many applications that can be used to diagnose themselves using this method. But they're all "passive" applications which you have to turn on the app and take pictures.
\\
\\The Point of this project lies in "active". 
Home appliances, typically refrigerators and TVs, are used by facing with users. Especially in the case of refrigerators, we know empirically that everyone opens it once a day, even if there's no specific reason. According to a 2012 Consumer Electronics Industry Survey, a family of four opens the refrigerator on average 40 times a day and it means 10 times a day per person.
Paying attention to this, \cite{r6} LG Electronics not only achieved great results with the design of "Magic Space" but also significantly reduced electricity use at home. \cite{r7}In addition, Americans open refrigerators an average of 33 times a day, according to ENERGY STAR, a U.S. environmental protection agency.\\
\\We pay attention to these behaviors. The refrigerator is equipped with a camera module to recognize the facial expressions of the user when standing in front of the refrigerator. If stroke was detected, it informs the user right away.

Unbeknownst to us, the disease is diagnosed and even follows up. Active appliances protect our healthy lives in real-time. If this technology is applied to all home appliances, The house can no longer be a space for living, but a diagnostic center for each individual.
You're worried about privacy? We've already prepared a countermeasure for this-.\\
\\Our project is UP version of refrigerator.\\
This is Innovation for a better life.
It makes Life to be Good.
\end{abstract}

%\begin{IEEEkeywords}
%component, formatting, style, styling, insert
%\end{IEEEkeywords}

\begin{table}[h]
    \caption{a list of role assignment}
    \begin{tabular}{|c|c|p{3.0cm}|}
    \hline
    Roles & Name & Task Description \\ \hline
      Development Manager    & Lee Seungsu    &  Lee Seungsu was responsible for designing the overall concept of the software and finding the basis for various claims.         \\ \hline
      Software Development    & Park Geonryul  &  Park Geonryul was in charge of the actual implementation of systems and machine learning. In addition, a basic framework for document creation was created. \\ \hline
        Customer  &   Elia Ayoub   &  As a customer, Elia uses the designed products and services produced by the team and delivers an objective feedback accordingly.          \\ \hline
        Document Analysis  &    Ryan Jabbour  &   Ryan was given the task of checking all the documentation to present and the logical development of the project.            \\ \hline
    \end{tabular}
    \end{table}

\section{Introduction}
\subsection{Motivation}
\begin{itemize}
    \item Problem: Unannounced emergencies, poor initial response \\
    \cite{r8}According to the National Statistical Office of the Republic of Korea, nearly 60,000 of the total 120,000 stroke patients in 2021 were not transferred to the emergency room until more than six hours after the outbreak.
Fewer than 15\% of the people arrived at the emergency room in less than an hour, and half of them were patients living in the Seoul-Gyeonggi area.
\cite{r9} The number of patients at risk increases in the provinces, but the medical infrastructure is insufficient, so initial diagnosis or prevention is not possible, and even if it could be, follow-up will inevitably be delayed.
After all, time is the lifeblood of a stroke. In order to do that, you need to quickly notice the signs. But it's rare to check for signs of stroke "every day".
We think the capture in everyday life is key. To do this, we'll have to mount active detectors on the ones we face most frequently.
Additionally, it is not just a project for the elderly. \cite{r8} According to the National Medical Center, the incidence of stroke among people in their 20s and 30s is rising every year.
Because of the stereotype that stroke is the disease of the elderly, even if you have premonitory symptoms, you will often pass it on, thinking of it as another reason.
\cite{r10} As the proportion of single-person households in Korea approaches 34.5\%, chances for them to recognize acute diseases in the early stages are also decreasing. 
In order to become a better society, we must overcome this.

The reality is that people's recognition rate of early stroke symptoms is also very low.
\cite{r11} According to the Korea Centers for Disease Control and Prevention, only 54\% of all respondents were correct for early stroke symptoms.
Although awareness reached a high of 61\% during the pandemic in 2019 - presumably because people cared a lot about health issues due to pandemic.
It has been low since 2019.
Even though we are living in an information society, this figure is that we don't usually think about stroke, and now you can see that why the effectiveness of passive stroke detection is very low.

The problems we thought of are summarized as follows.

First, people aren't as wary or concerned about strokes as we might think.
Second, when a stroke occurs, it is quite rare to arrive at a hospital within an hour, the golden time, and if it can, most of them live in the metropolitan area with medical infrastructure that allows immediate emergency action.
Third, the elderly, who are at high risk of getting the disease, are concentrated in the provinces where have no infrastructure, and have low awareness, so it is unlikely to respond to premonitory symptoms.
Fourth, although a situation where the incidence rate in their 20s and 30s is rising, their vigilance is very low.
Fifth, as the number of single-person households increases, there are fewer opportunities for them to recognize and initial responses are becoming insufficient.

In order for the passive detector to be effective, people of all ages must be aware of stroke on their own and check it periodically. So the effect of it does not seem to be able to be enhanced by any method-promotion, campaign, etc. -If this could be elevated, We think \cite{r12}the prognostic indicator for stroke should have been more positive.

Therefore, what we need is an "active daily checker", which we think can be made through home appliances, typically refrigerators or TVs.
    
\end{itemize}
\\
\begin{itemize}
    \item Solve\\
    As I said at the beginning, \cite{r7} We open the refrigerator once a day for a reason or not. Our concept uses this habit.
There are already many refrigerators equipped with IoT technology. We're going to turn up our home appliances by locating a camera module here.
The refrigerator detects the person's face and its landmark through computer vision when a person stands in front. 
Among BE-FAST diagnostic methods, if facial expressions such as paralysis of one facial muscle, are detected, stroke-striker notify it right away.
The notification method is that if you have a refrigerator with a display, it can tell you through the display, and if you have not, it can tell you through a push-message on your phone application (Thin-Q) or use AI speaker (NUGU).
At a later stage, it will guide you to the nearest hospital where first aid for stroke is available, and automatically connect to 119 if the user wants.
Since fixed cameras may not respond appropriately depending on the user's physical characteristics, \cite{r13} multi-angle vision technology is applied to detect users from various angles.
This creates a true daily active detector beyond the limits of different physical and home structural characteristics for each user.
\end{itemize}
\\
\begin{itemize}
    \item Expectation \\
    As the artificial intelligence field is rapidly developing and computer vision is a technology that occupies a large proportion of it, it is highly likely to detect user behavior or face and develop it into various medical diagnosis.
If these technologies are included in each of LG Electronics' home appliances, which currently have a huge share of home appliances compared to competitors, users will be able to continue to actively observe them in their homes, whether in living rooms, kitchens or bedrooms.
This will allow the home to become a diagnostic center for individuals that move actively beyond just living spaces, and if this becomes a reality, we expect a very big paradigm shift.
\cite{r14}We think home diagnostics self care, which is developing recently, is a very important technology field, and the synergy will be great if it is combined with home appliances.
\end{itemize}

\subsection{Research on Related Materials}
\begin{itemize}
    \item Project MONAI \\
    MONAI is an initiative started by NVIDIA and King's College London to establish an inclusive community of AI researchers to develop and exchange best practices for AI in healthcare. This collaboration has expanded to include academic and industry leaders throughout the medical field. \\
    This project is similar to our project because it is simply analyzing MRI or CT photographs with AI, but the methods used are different.
    \item BASLER \\
    % 비전 시스템의 전반적인 솔루션을 제공한다. 하드웨어와 소프트웨어를 동시에 지원하고 머신러닝 기반으로 이미지를 판별할 수 있다. 특히, 의료 관련에 특화되어 있다. 하지만 센서와 카메라가 매우 비싸 그대로 가전에 적용하기는 어렵다.
    This project provides an overall solution for the vision system. It supports hardware and software at the same time and can analyse images based on machine learning. It is specialized in medical care particularly. However, sensors and cameras are very expensive, so it would be difficult to apply them to home appliances as they are presented in this project.
    \item Kaggle Project \\
    % kaggle에서 진행하는 뇌졸중 감지 프로젝트이다. 우리 프로젝트의 AI 모델로써 사용가능하다. 이 프로젝트의 판별 알고리즘은 2D 이미지를 기반으로 진행하기 때문에 우리 프로젝트의 3D 인식과는 다른 점이 있다.
    It is a stroke detection project undertaken by Kaggle. It can be used as an AI model for our project but since the algorithm used in this project is based on 2D images, it differs from the 3D recognition we need to use in our project.
    \item Related Papers \\
    % 우리의 프로젝트의 이론적인 근거를 찾기 위해 많은 논문을 조사했다. 조사한 결과 주요한 논문은 다음과 같다. FAST 기법으로 얼굴 사진으로 뇌졸중을 판단할 수 있고, 하나의 카메라로 multi-angle의 얼굴 표정을 검출할 수 있다. 마지막으로 CNN을 이용해 facial landmark로 딥러닝을 수행할 수 있음을 알 수 있었다.
    We researched a number of papers to find the theoretical part for our project.\\ 
    % The FAST technique can determine a stroke with a facial photo.Then, a multi-angle facial expression can be detected with one camera. Finally, we found that deep learning can be performed with facial landmark using CNN.

    1. Multi-Angle detector\cite{r15}
    \\
    \\This paper introduce lightweight deep network and combining key point feature positioning for multi-angle face expression recognition. Using robot dog to recognize facial expressions will be affected by distance and angle. To solve this problem, this paper proposes a method for facial expression recognition at different distances and angles,which solved the larger distance and deflection angle of face expression recognition \\

    2. Raspberry Pi Based Emotion Recognition using OpenCV, TensorFlow, and Keras \cite{r16}
    \\
    \\It implement an Emotion Recognition System or a Facial Expression Recognition System on a Raspberry Pi 4. It apply a pre-trained model to recognize the facial expression of a person from a real-time video stream. The “FER2013” dataset is used to train the model with the help of a VGG-like Convolutional Neural Network (CNN).\\

    3. Connect a Raspberry Pi or other device with AWS\cite{r17} \\
    \\It tells how to Set up device, install the required tools and libraries for the AWS IoT Device SDK, install AWS IoT Device SDK, install and run the sample appView messages from the sample app in the AWS IoT console.\\

    4. Realtime Facial Emotion Recognition\cite{r18} \\
    \\This repository demonstrates an end-to-end pipeline for real-time Facial emotion recognition application through full-stack development. The front-end is developed in react.js and the back-end is developed in FastAPI. The emotion prediction model is built with Tensorflow Keras, and for real-time face detection with animation on the frontend, Tensorflow.js have been used.\\

    5. Kaggle FER-2013 DataSet\cite{r19} \\
    \\The data consists of 48x48 pixel grayscale images of faces. The faces have been automatically registered so that the face is more or less centred and occupies about the same amount of space in each image.
The task is to categorize each face based on the emotion shown in the facial expression into one of seven categories (0=Angry, 1=Disgust, 2=Fear, 3=Happy, 4=Sad, 5=Surprise, 6=Neutral). The training set consists of 28,709 examples and the public test set consists of 3,589 examples.\\

    6. Facial landmarks with dlib, OpenCV, and Python\cite{r20} \\
    \\This document tell What are face landmarks, Understanding dlib’s facial landmark detector, How Detect facial landmarks with dlib, OpenCV, and Python,
    how visualization facail landmarks with co-lab,Google.
    Also, it introduce alternative facial landmark detectors, such as MediaPipe library which is capable of computing a 3D face mesh.\\
    
\end{itemize}

\section{Requirements}
\subsection{AI model}
% 사용자의 사진을 입력받아 뇌졸중의 여부를 판단하는 인공지능이다. 사람의 얼굴 사진으로 미리 학습을 시킨다. 이후 웹 서버에 배포를 해서 API를 통해 사진을 보내면 인식할 수 있다. 추론을 학습된 데이터를 기반으로 진행해 결괏값을 다시 API로 전송해 임의의 사람의 얼굴에 대한 뇌줄중 여부를 Face drooping을 근거로 판단하는 인공지능 모델이다.
It is an artificial intelligence that determines whether a stroke occurs by receiving a user's photo. Learn in advance with pictures of people's faces. Afterwards, you can recognize it by distributing it to a web server and sending a photo through the API. It is an artificial intelligence model that conducts reasoning based on learned data and transmits the result value back to the API to determine whether or not a brain line is drawn to a person's face based on Face drooping.

\subsection{Web communication}
% 웹 통신을 위해 다음과 같은 기능이 필요하다. AWS에 존재하는 인공지능 모델과 라즈베리 파이 사이의 통신을 지원한다. key, value로 이루어진 json 포맷으로 통신이 이루어지며 POST 방식으로 prdict가 이루어진다. 
It is an API for web communication. This project supports communication between AWS' existing artificial intelligence model and Raspberry Pie. Communication is carried out in the json format consisting of key and value, and the prediction is carried out in the POST method.
\subsubsection{Post}
% 웹으로 Post 할 때는 사용자의 이미지가 전달된다. 이미지가 Post 명령에 의해 전달되고 인공지능 모델이 인식할 수 있다. 인공지능 모델이 인식할 수 있도록 이미지를 변환하는 과정은 구현의 일관성을 유지하기 위해 아래의 handler에서 실행된다.
When posting on the web, the user's image is delivered. The image is delivered by post command and recognizable by the artificial intelligence model. The process of converting images so that artificial intelligence models can recognize them is executed in the handler below to maintain consistency in implementation.
\subsubsection{Get}
% 학습된 인공지능 모델이 Post된 자료를 받아 처리한 후 예측값을 다시 Raspberry Pi로 전해준다. 이 때, 값은 뇌졸중 가능성을 나타내는 확률값으로 정의된다. 단순히 0 또는 1로 판별하기에는 위험이 크다고 판단했기 때문이다.
The learned artificial intelligence model receives and processes the posted data and delivers the predicted value back to Raspberry Pi. In this case, the value is defined as a probability value indicating a stroke probability. This is because it was judged that there was a high risk to simply distinguish it as 0 or 1.

\subsection{AWS}
% 인공지능 모델의 배포와 학습을 위한 서버이다. Ubuntu 기반의 x86-64로 구성되어 있다. EC2를 사용해 가상 인스턴스를 만들었으며 탄력적 IP로 접근 가능하도록 보안 설정을 했다.
It is a server for distribution and learning of artificial intelligence models. It consists of Ubuntu-based x86-64. The virtual instance was created using EC2 and security settings were set to be accessible with Elastic IP.

\subsection{Raspberry Pi}
% 사용자의 사진을 촬영하는 카메라를 통제하고 촬영한 사진을 웹서버와 통신이 가능하게 한다. 인공지능 모델이 사진을 기반으로 뇌졸중 여부를 판단한 결과를 다시 받아 다시 사용자에게 알려주는 역할을 맡았다. 이 과정은 NUGU 스피커로 청각적으로나 Dashboard를 통해 시각적으로 표현이 가능하다.
It controls the camera that takes a user's picture and enables communication with the web server. The artificial intelligence model took on the role of receiving the results of determining whether a stroke occurred based on pictures and notifying users again. This process can be expressed both audibly and visually through the Dashboard with NUGU speakers.

\subsection{Tensorflow Serving}
% Tensorflow를 기반으로 인공지능 모델을 개발했을 때, 배포를 간편하게 진행할 때 필요하다.
When an artificial intelligence model is developed based on Tensorflow, it is necessary to easily proceed with distribution.

\subsection{Handler}
% 학습된 인공지능 모델이 웹 서버에 존재할 때 전송되는 이미지 자료를 판단할 수 있도록 변형하고 전처리하는 과정과 예측한 결괏값을 API를 통해 돌려주기 위해 문자열을 제작하는 과정의 집합체이다. 이 기능을 통해 인공지능 모델의 배포와 유지관리를 쉽게 할 수 있다.
It is a collection of the process of transforming and preprocessing the image data transmitted when the learned artificial intelligence model exists on the web server and the process of producing strings to return the predicted result value through API. This function makes it easy to distribute and maintain artificial intelligence models.

\subsection{Disable camera}
% 사용자의 사진을 촬영하는 카메라로서 사용되지 않는 경우에는 작동이 불가능해야 한다. 개인정보 보호를 위해 카메라가 작동 중이면 맥북의 웹캠처럼 표시를 가능하게 하거나 zoom의 기능과 같이 주변부를 흐리게 처리할 수 있다.
If it is not used as a camera to take a user's picture, it should be impossible to operate. If the camera is working for privacy protection, it can be displayed like a webcam on a MacBook or cloud the surroundings like a Zoom function.

\subsection{NUGU}
% 인공지능 스피커로 본 프로젝트에서는 라즈베리 파이에서 전송된 뇌졸중 여부를 사용자에게 청각적으로 전달할 수 있다. 이 외에 사용자가 원할 때 촬영을 시작하도록 trigger를 인식할 수 있다.
In this project with an artificial intelligence speaker, it is possible to aurally transmit the stroke transmitted from the raspberry pie to the user. In addition, the trigger may be recognized so that the user starts photographing when desired.

\subsection{Dashboard}
% 뇌졸중 여부를 시각적으로 표현해준다. 라즈베리 파이에서 전송된 뇌졸중 여부와 확률, chatGPT로 작성한 건강에 관련된 유용한 정보를 사용자에게 시각적으로 보여준다. 유용한 정보의 예시는 다음과 같다. 뇌졸중에 도움이 되는 음식이나 생활습관, 만약 발병한다면 어떻게 해야 하는지에 대한 정보가 포함된다.
It visually expresses whether you have a stroke. Visually shows the user useful information related to stroke and probability transmitted from raspberry pie, and health written by chatGPT. Examples of useful information are as follows. It includes information on food or lifestyle habits that help with a stroke, and what to do if it develops.
\subsubsection{Possibility}
% API를 통해 전송된 값에서 뇌졸중 여부의 확률값을 표현한다. 자동차의 속도 계기판을 모티브로 하여 사용자에게 직관적으로 확률을 전달한다.
It expresses the probability value of stroke from the value transmitted through the API. It intuitively transmits probabilities to users by using the speed dashboard of the car as a motif.
\subsubsection{User's image}
% 사용자를 촬영한 사진을 보여준다. 이로써 사용자는 자신의 상태를 객관적으로 확인할 수 있으며 경각심을 일으킬 수 있다.
Shows a photo of the user. As a result, the user can objectively check his or her state.
\subsubsection{Cure}
% 아래의 chatGPT를 통해 얻은 뇌졸중의 치료법을 보여주는 공간이다. 만약 사용자가 뇌졸중의 확률이 높을 경우에는 더욱 눈에 띄도록 표현할 수 있다.
This is a space showing the treatment of stroke obtained through chatGPT below. If the user has a high probability of stroke, it can be expressed more prominently.
\subsubsection{Preventive}
% 아래의 chatGPT를 통해 얻은 뇌졸중의 예방법을 알려준다. 뇌졸중의 확률이 낮더라도 예방법을 사용자에게 알려주고, 만약 의심될 때는 어떻게 해야하는지 행동지침을 알린다. 이로써 능동적인 건강관리를 가능케 한다. 
It tells users how to prevent stroke obtained through chatGPT below. Even if the probability of stroke is low, it informs the user of prevention and informs the behavioral guidelines on what to do if suspected. This enables active health care.

\subsection{chatGPT}
% 사용자의 뇌졸중 여부와 확률을 기반으로 유용한 정보를 생성해주는 생성형 AI이다. API를 이용해서 질문을 담아서 전송하고 답변을 다시 받아온다. 그 답변을 Dashboard 혹은 NUGU를 통해 사용자에게 알려준다. 
It is a generative AI that generates useful information based on the user's stroke status and probability. It uses API to send questions and receive answers again. The answer is notified to the user through the Dashboard or NUGU.
\subsubsection{Question}
% chatGPT에게 뇌졸중 관련으로 질문할 때 피상적인 표현으로 한다면 당장 병원에 가야한다는 정보만 얻는다. 인공지능 모델이 반환한 결괏값에서 확률을 추출하고 치료법과 예방법, 병원을 어떻게 골라야하는지를 직접적으로 물어본다. 
When I ask chatGPT about stroke, we only get information that I have to go to the hospital right away if we do it in a superficial way. It extracts probabilities from the results returned by the artificial intelligence model and directly asks how to choose treatments, prevention, and hospitals.
\subsubsection{Answer}
% 위의 질문 형식으로 chatGPT에게 질문하고 받은 대답이다. 이 대답은 Dashboard 혹은 NUGU로 전달되어 각각 시각적, 청각적인 표현으로 사용자에게 전달된다.
This is the answer I received after I asked the chatGPT in the above question format. This answer is passed to the Dashboard or NUGU and is passed to the user in visual and auditory expressions, respectively.

\subsection{Database}
% 인공지능 모델의 정확도를 향상시키기 위해서 이미지를 보관하는 데이터베이스가 필요할 수 있다. 하지만 이는 개인정보로 민감할 수 있기에 데이터베이스 없이 프로젝트를 구현하는 것으로 진행할 수도 있다.
In order to improve the accuracy of the artificial intelligence model, a database for storing images may be required. However, since this can be sensitive to personal information, the project can be implemented without a database.





% \section{Ease of Use}

% \subsection{Maintaining the Integrity of the Specifications}

% The IEEEtran class file is used to format your paper and style the text. All margins, 
% column widths, line spaces, and text fonts are prescribed; please do not 
% alter them. You may note peculiarities. For example, the head margin
% measures proportionately more than is customary. This measurement 
% and others are deliberate, using specifications that anticipate your paper 
% as one part of the entire proceedings, and not as an independent document. 
% Please do not revise any of the current designations.

% \section{Prepare Your Paper Before Styling}
% Before you begin to format your paper, first write and save the content as a 
% separate text file. Complete all content and organizational editing before 
% formatting. Please note sections \ref{AA}--\ref{SCM} below for more information on 
% proofreading, spelling and grammar.

% Keep your text and graphic files separate until after the text has been 
% formatted and styled. Do not number text heads---{\LaTeX} will do that 
% for you.

% \subsection{Abbreviations and Acronyms}\label{AA}
% Define abbreviations and acronyms the first time they are used in the text, 
% even after they have been defined in the abstract. Abbreviations such as 
% IEEE, SI, MKS, CGS, ac, dc, and rms do not have to be defined. Do not use 
% abbreviations in the title or heads unless they are unavoidable.

% \subsection{Units}
% \begin{itemize}
% \item Use either SI (MKS) or CGS as primary units. (SI units are encouraged.) English units may be used as secondary units (in parentheses). An exception would be the use of English units as identifiers in trade, such as ``3.5-inch disk drive''.
% \item Avoid combining SI and CGS units, such as current in amperes and magnetic field in oersteds. This often leads to confusion because equations do not balance dimensionally. If you must use mixed units, clearly state the units for each quantity that you use in an equation.
% \item Do not mix complete spellings and abbreviations of units: ``Wb/m\textsuperscript{2}'' or ``webers per square meter'', not ``webers/m\textsuperscript{2}''. Spell out units when they appear in text: ``. . . a few henries'', not ``. . . a few H''.
% \item Use a zero before decimal points: ``0.25'', not ``.25''. Use ``cm\textsuperscript{3}'', not ``cc''.)
% \end{itemize}

% \subsection{Equations}
% Number equations consecutively. To make your 
% equations more compact, you may use the solidus (~/~), the exp function, or 
% appropriate exponents. Italicize Roman symbols for quantities and variables, 
% but not Greek symbols. Use a long dash rather than a hyphen for a minus 
% sign. Punctuate equations with commas or periods when they are part of a 
% sentence, as in:
% \begin{equation}
% a+b=\gamma\label{eq}
% \end{equation}

% Be sure that the 
% symbols in your equation have been defined before or immediately following 
% the equation. Use ``\eqref{eq}'', not ``Eq.~\eqref{eq}'' or ``equation \eqref{eq}'', except at 
% the beginning of a sentence: ``Equation \eqref{eq} is . . .''

% \subsection{\LaTeX-Specific Advice}

% Please use ``soft'' (e.g., \verb|\eqref{Eq}|) cross references instead
% of ``hard'' references (e.g., \verb|(1)|). That will make it possible
% to combine sections, add equations, or change the order of figures or
% citations without having to go through the file line by line.

% Please don't use the \verb|{eqnarray}| equation environment. Use
% \verb|{align}| or \verb|{IEEEeqnarray}| instead. The \verb|{eqnarray}|
% environment leaves unsightly spaces around relation symbols.

% Please note that the \verb|{subequations}| environment in {\LaTeX}
% will increment the main equation counter even when there are no
% equation numbers displayed. If you forget that, you might write an
% article in which the equation numbers skip from (17) to (20), causing
% the copy editors to wonder if you've discovered a new method of
% counting.

% {\BibTeX} does not work by magic. It doesn't get the bibliographic
% data from thin air but from .bib files. If you use {\BibTeX} to produce a
% bibliography you must send the .bib files. 

% {\LaTeX} can't read your mind. If you assign the same label to a
% subsubsection and a table, you might find that Table I has been cross
% referenced as Table IV-B3. 

% {\LaTeX} does not have precognitive abilities. If you put a
% \verb|\label| command before the command that updates the counter it's
% supposed to be using, the label will pick up the last counter to be
% cross referenced instead. In particular, a \verb|\label| command
% should not go before the caption of a figure or a table.

% Do not use \verb|\nonumber| inside the \verb|{array}| environment. It
% will not stop equation numbers inside \verb|{array}| (there won't be
% any anyway) and it might stop a wanted equation number in the
% surrounding equation.

% \subsection{Some Common Mistakes}\label{SCM}
% \begin{itemize}
% \item The word ``data'' is plural, not singular.
% \item The subscript for the permeability of vacuum $\mu_{0}$, and other common scientific constants, is zero with subscript formatting, not a lowercase letter ``o''.
% \item In American English, commas, semicolons, periods, question and exclamation marks are located within quotation marks only when a complete thought or name is cited, such as a title or full quotation. When quotation marks are used, instead of a bold or italic typeface, to highlight a word or phrase, punctuation should appear outside of the quotation marks. A parenthetical phrase or statement at the end of a sentence is punctuated outside of the closing parenthesis (like this). (A parenthetical sentence is punctuated within the parentheses.)
% \item A graph within a graph is an ``inset'', not an ``insert''. The word alternatively is preferred to the word ``alternately'' (unless you really mean something that alternates).
% \item Do not use the word ``essentially'' to mean ``approximately'' or ``effectively''.
% \item In your paper title, if the words ``that uses'' can accurately replace the word ``using'', capitalize the ``u''; if not, keep using lower-cased.
% \item Be aware of the different meanings of the homophones ``affect'' and ``effect'', ``complement'' and ``compliment'', ``discreet'' and ``discrete'', ``principal'' and ``principle''.
% \item Do not confuse ``imply'' and ``infer''.
% \item The prefix ``non'' is not a word; it should be joined to the word it modifies, usually without a hyphen.
% \item There is no period after the ``et'' in the Latin abbreviation ``et al.''.
% \item The abbreviation ``i.e.'' means ``that is'', and the abbreviation ``e.g.'' means ``for example''.
% \end{itemize}
% An excellent style manual for science writers is \cite{b7}.

% \subsection{Authors and Affiliations}
% \textbf{The class file is designed for, but not limited to, six authors.} A 
% minimum of one author is required for all conference articles. Author names 
% should be listed starting from left to right and then moving down to the 
% next line. This is the author sequence that will be used in future citations 
% and by indexing services. Names should not be listed in columns nor group by 
% affiliation. Please keep your affiliations as succinct as possible (for 
% example, do not differentiate among departments of the same organization).

% \subsection{Identify the Headings}
% Headings, or heads, are organizational devices that guide the reader through 
% your paper. There are two types: component heads and text heads.

% Component heads identify the different components of your paper and are not 
% topically subordinate to each other. Examples include Acknowledgments and 
% References and, for these, the correct style to use is ``Heading 5''. Use 
% ``figure caption'' for your Figure captions, and ``table head'' for your 
% table title. Run-in heads, such as ``Abstract'', will require you to apply a 
% style (in this case, italic) in addition to the style provided by the drop 
% down menu to differentiate the head from the text.

% Text heads organize the topics on a relational, hierarchical basis. For 
% example, the paper title is the primary text head because all subsequent 
% material relates and elaborates on this one topic. If there are two or more 
% sub-topics, the next level head (uppercase Roman numerals) should be used 
% and, conversely, if there are not at least two sub-topics, then no subheads 
% should be introduced.

% \subsection{Figures and Tables}
% \paragraph{Positioning Figures and Tables} Place figures and tables at the top and 
% bottom of columns. Avoid placing them in the middle of columns. Large 
% figures and tables may span across both columns. Figure captions should be 
% below the figures; table heads should appear above the tables. Insert 
% figures and tables after they are cited in the text. Use the abbreviation 
% ``Fig.~\ref{fig}'', even at the beginning of a sentence.

% \begin{table}[htbp]
% \caption{Table Type Styles}
% \begin{center}
% \begin{tabular}{|c|c|c|c|}
% \hline
% \textbf{Table}&\multicolumn{3}{|c|}{\textbf{Table Column Head}} \\
% \cline{2-4} 
% \textbf{Head} & \textbf{\textit{Table column subhead}}& \textbf{\textit{Subhead}}& \textbf{\textit{Subhead}} \\
% \hline
% copy& More table copy$^{\mathrm{a}}$& &  \\
% \hline
% \multicolumn{4}{l}{$^{\mathrm{a}}$Sample of a Table footnote.}
% \end{tabular}
% \label{tab1}
% \end{center}
% \end{table}

% \begin{figure}[htbp]
% \centerline{\includegraphics{}}
% \caption{Example of a figure caption.}
% \label{fig}
% \end{figure}

% Figure Labels: Use 8 point Times New Roman for Figure labels. Use words 
% rather than symbols or abbreviations when writing Figure axis labels to 
% avoid confusing the reader. As an example, write the quantity 
% ``Magnetization'', or ``Magnetization, M'', not just ``M''. If including 
% units in the label, present them within parentheses. Do not label axes only 
% with units. In the example, write ``Magnetization (A/m)'' or ``Magnetization 
% \{A[m(1)]\}'', not just ``A/m''. Do not label axes with a ratio of 
% quantities and units. For example, write ``Temperature (K)'', not 
% ``Temperature/K''.

% \section*{Acknowledgment}

% The preferred spelling of the word ``acknowledgment'' in America is without 
% an ``e'' after the ``g''. Avoid the stilted expression ``one of us (R. B. 
% G.) thanks $\ldots$''. Instead, try ``R. B. G. thanks$\ldots$''. Put sponsor 
% acknowledgments in the unnumbered footnote on the first page.

% \section*{References}

% Please number citations consecutively within brackets \cite{b1}. The 
% sentence punctuation follows the bracket \cite{b2}. Refer simply to the reference 
% number, as in \cite{b3}---do not use ``Ref. \cite{b3}'' or ``reference \cite{b3}'' except at 
% the beginning of a sentence: ``Reference \cite{b3} was the first $\ldots$''

% Number footnotes separately in superscripts. Place the actual footnote at 
% the bottom of the column in which it was cited. Do not put footnotes in the 
% abstract or reference list. Use letters for table footnotes.

% Unless there are six authors or more give all authors' names; do not use 
% ``et al.''. Papers that have not been published, even if they have been 
% submitted for publication, should be cited as ``unpublished'' \cite{b4}. Papers 
% that have been accepted for publication should be cited as ``in press'' \cite{b5}. 
% Capitalize only the first word in a paper title, except for proper nouns and 
% element symbols.

% For papers published in translation journals, please give the English 
% citation first, followed by the original foreign-language citation \cite{b6}.

% \begin{thebibliography}{00}
% \bibitem{1} National Statistical Office, Cause of death statistics results, 2022

% \bibitem{2} About Stroke, American Stroke Association, http://aiweb.techfak.uni-bielefeld.de/content/bworld-robot-control-software/

% \bibitem{3} Spiotta, Alejandro M., et al. "The golden hour of stroke intervention: effect of thrombectomy procedural time in acute ischemic stroke on outcome." Journal of neurointerventional surgery 6.7 (2014): 511-516.

% \bibitem{4} 뇌졸중, 위키백과, https://ko.wikipedia.org/w/index.phptitle=%EB%87%8C%EC%A1%B8%EC%A4%91&oldid=34409902 

% \bibitem{5} El Ammar, F., Ardelt, A., Del Brutto, V. J., Loggini, A., Bulwa, Z., Martinez, R. C., ... & Goldenberg, F. D. (2020). BE-FAST: a sensitive screening tool to identify in-hospital acute ischemic stroke. Journal of stroke and cerebrovascular diseases, 29(7), 104821.

% \bibitem{6} 중앙일보, 60\% 더 비싼 ‘문안의 문’ 냉장고 불티 왜, https://www.joongang.co.kr/article/8013466#home, 2012

% \bibitem{7} ELIZABETHTOWN GAS, Cold Facts About Your Refrigerator, https://www.elizabethtowngas.com/conserve/cold-facts-about-your-refrigerator

% \bibitem{8} Leesungsu
% \end{thebibliography}
% \vspace{14pt}
% \color{red}
% IEEE conference templates contain guidance text for composing and formatting conference papers. Please ensure that all template text is removed from your conference paper prior to submission to the conference. Failure to remove the template text from your paper may result in your paper not being published.

\bibliographystyle{unsrt}
\bibliography{reference}


\end{document}
